\documentclass[10pt,a4paper,UTF8]{ctexart}
\usepackage{geometry}%用于设置上下左右页边距
	\geometry{left=2.5cm,right=2.5cm,top=3.2cm,bottom=2.8cm}
\usepackage{xeCJK,amsmath,paralist,enumerate,booktabs,multirow,graphicx,subfig,setspace,listings,lastpage,hyperref}
\usepackage{amsthm, amssymb, bm, color, framed, graphicx, hyperref, mathrsfs}
\usepackage{mathrsfs}  
	\setlength{\parindent}{2em}
	\lstset{language=Matlab}%
\usepackage{fancyhdr}
\usepackage{graphicx}
\usepackage{listings}
\usepackage{xcolor}
\usepackage{float}

\definecolor{mKeyword}{RGB}{0,0,255}          % bule
\definecolor{mString}{RGB}{160,32,240}        % purple
\definecolor{mComment}{RGB}{34,139,34}        % green
\definecolor{mNumber}{RGB}{128,128,128} 

\lstdefinestyle {njulisting} {
	basewidth = 0.5 em,
	lineskip = 3 pt,
	basicstyle = \small\ttfamily,
	% keywordstyle = \bfseries,
	commentstyle = \itshape\color{gray}, 
	basicstyle=\small\ttfamily,
	keywordstyle={\color{mKeyword}},     % sets color for keywords
	stringstyle={\color{mString}},       % sets color for strings
	commentstyle={\color{mComment}},     % sets color for comments
	numberstyle=\tiny\color{mNumber},
	numbers = left,
	captionpos = t,
	breaklines = true,
	xleftmargin = 2 em,
	xrightmargin = 2 em,
	frame=tlrb,
	tabsize=4
}

\lstset{
style = njulisting, % 调用上述样式 
flexiblecolumns % 允许调整字符宽度
}

\pagestyle{fancy}
\lhead{\textsc{Foundation of Computing System}}
\rhead{\textsc{Nanjing University}}
\cfoot{\thepage}
\renewcommand{\headrulewidth}{0.4pt}
\renewcommand{\theenumi}{(\arabic{enumi})}


\definecolor{shadecolor}{RGB}{241, 241, 255}

\newcommand{\problemname}{待定义}
\newenvironment{problem}{\begin{shaded}\par\noindent\textbf{题目\  \problemname}}{\end{shaded}\par}
\newenvironment{solution}{\par\noindent\textbf{解答}\ }{\par}
\newenvironment{note}{\par\noindent\textbf{题目 \problemname 的注记}}{\par}

\begin{document}

\begin{center}
\LARGE\textbf{第十三章习题参考答案}
\end{center}

{\kaishu 包含题目:习题$13.3$、$13.4$、$13.5$、$13.7$、$13.10$、$13.13$和$13.14$}


% \begin{figure}[H]
% 	\centering
% 	\includegraphics[scale=0.3]{img/}
% \end{figure}

% \lstset{language=C}
% 	\begin{lstlisting}
	
% 	\end{lstlisting}



\renewcommand{\problemname}{13.3}
\begin{problem}
	设计一个新的TRAP服务例程,该服务例程读入一个字符并回显到屏幕上,如下所示:
	\begin{lstlisting}
			.data	x00003C00 
SaveR31:	.space	4  
;		 
			.text	x0000 3D00 
			sw	SaveR31(r0), r31 
			getc 
			out 
; 
			lw	r31, SaveR31(r0) 
			jr	r31 
	\end{lstlisting}
	\begin{enumerate}[(1)]
		\item 请说明调用此服务例程的指令。假设每个服务例程均占用$2^{10}$个单元。
		\item 此服务例程能否正常运行?
	\end{enumerate}
\end{problem}

\begin{solution}
	\begin{enumerate}[(1)]
		\item TRAP x08的数据段地址是x00002C00,这个服务例程的是x00003C00,相差$2^{12}$,
		每个服务例程占有$2^{10}$个单元,所以该服务的调用指令为TRAP x0C。
		\item \begin{itemize}
			\item 能够运行,getc和out是指令助记符。
			\item 不能够运行,没有对R4进行保存/恢复工作。
		\end{itemize}
	\end{enumerate}

\end{solution}


\renewcommand{\problemname}{13.4}
\begin{problem}
	重新定义DLX的TRAP机制,使其不使用TRAP向量表:
	\begin{enumerate}[(1)]
		\item 通过将TRAP向量左移8位,形成相应服务例程的起始地址,那么,
		每个服务例程最多可以占用多少存储单元?
		256个TRAP服务例程需要占用多少存储单元?
		\item TRAP向量提供的就是相应服务例程的起始地址,每个服务例程占用$2^{10}$个单元,
		那么,256个TRAP服务例程需要占用多少存储单元? 
	\end{enumerate}
\end{problem}

\begin{solution}
	\begin{enumerate}[(1)]
		\item TRAP x08对应的服务例程起始地址是x00000800,
		TRAP x09对应的服务例程起始地址是x00000900,因此,每个服务例程最多有$2^8$个单元;
		256个TRAP服务例程需要占用$2^{16}$个存储单元。
		\item 需要占用$2^{18}$个存储单元
	\end{enumerate}
\end{solution}


\renewcommand{\problemname}{13.5}
\begin{problem}
	执行如下程序,出现了一些异常行为,请解释原因。
	\begin{lstlisting}
		.text		x4000 0000 
		.global		main 
main:	SW			24(R0), R0 
		GETC 
		OUT 
		TRAP		x00 	
	\end{lstlisting}
\end{problem}

\begin{solution}
	内存单元x0000 0000 $\sim$ x0000 03FF被用来存储TRAP向量表,不能对其进行写入操作。
\end{solution}


\renewcommand{\problemname}{13.7}
\begin{problem}
	执行如下汇编语言程序,输出是什么?
	\begin{lstlisting}
			.data		x30000000 
Char:		.word		x61626364 
HelloWorld:	.asciiz  	"Hello, World!" 
;		 
			.text		x40000000 
			.global		main 
main:		addi		r1, r0, #0 
			addi		r4, r0, Char 
			sw			4(r4), r1 
			trap		x08 
; 
Exit:		trap		x00 
	\end{lstlisting}
\end{problem}

\begin{solution}
	屏幕输出“abcd”
\end{solution}


\renewcommand{\problemname}{13.10}
\begin{problem}
	中断驱动的I/O
	\begin{enumerate}
		\item 如下程序实现了什么?
		\begin{lstlisting}
		.data	x30000000 
KBSR:	.word	xFFFF0000 
; 
		.text	x40000000 
		.global	main 
main:	lw		r1, KBSR(r0) 
		lw		r2, 0(r1) 
		ori		r2, r2, #2 
		sw		0(r1), r2 
LOOP:	addi	r4, r0, x41 
		out 
		j		LOOP 
		trap	x00 
		\end{lstlisting}
		\item 如果有人键入了一个键,程序将被中断。
		修改图13.11给出的中断服务例程,对键盘的处理片段修改如下:
		\begin{lstlisting}
DEV1:	lw	r1, KBDR(r0) 
		lw	r4, 0(r1)	 
		out					; 新增的内容
		jr	r7 
		\end{lstlisting}
		假设程序(1)开始执行,坐在键盘前的人键入了一个“8”,
		那么,屏幕上将显示什么内容?提示:“8”可于程序执行的任意时刻输入。 
	\end{enumerate}

\end{problem}

\begin{solution}
	\begin{enumerate}[(1)]
		\item 程序将KBSR[1]置1,并且不断输出“A”。
		\item 如果在原程序out指令前中断,则屏幕上将显示“A...A88A...A”;
				如果在原程序out指令后中断,则屏幕上将显示“A...A8A...A”。
	\end{enumerate}

\end{solution}


\renewcommand{\problemname}{13.13}
\begin{problem}
	请写出处理如下任务的C语言的I/O函数调用。每个任务只需要一次调用。
	\begin{enumerate}[(1)]
		\item 将一个电话号码按照(XX)-XX-XXXXXXXX的格式输出,假设电话号码被存储于3个整数变量中。
		\item 读取一个(XX)XX,XXXXXXXX格式的电话号码到3个整数变量中。
	\end{enumerate}
\end{problem}

\begin{solution}
	\begin{enumerate}[(1)]
		\item \ 
		\begin{lstlisting}[language=C]
int first, second, third;
…
printf ("The phone number is: (%d)-%d-%d\n", first, second, third);
		\end{lstlisting}
		\item \ 
		\begin{lstlisting}[language=C]
int first, second, third;
scanf ("(%d)%d,%d", &first, &second, &third);
		\end{lstlisting}

	\end{enumerate}

\end{solution}


\renewcommand{\problemname}{13.14}
\begin{problem}
	对于如下程序:
	\begin{lstlisting}[language=C]
#include <stdio.h> 

int main() { 
	int x = 0; 
	int y = 0; 
	char a = 'a'; 
	char b = 'b';
	a = getchar(); 
	scanf ("%d%d", &x, &y); 
	b = getchar(); 

	printf ("%d %d %c %c", x, y, a, b); 
} 

	\end{lstlisting}
	\begin{enumerate}[(1)]
		\item 如果输入流是10 20 C,程序的输出是什么?
		\item 如果输入流是C 10 20D,程序的输出是什么?
		\item 如果输入流是10 C,程序的输出是什么?
	\end{enumerate}
\end{problem}

\begin{solution}
	\begin{enumerate}[(1)]
		\item 0 20 1 [空格]
		\item 10 20 C D
		\item 0 0 1 C
	\end{enumerate}
\end{solution}

\end{document}
