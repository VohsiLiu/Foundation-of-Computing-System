\documentclass[10pt,a4paper,UTF8]{ctexart}
\usepackage{geometry}%用于设置上下左右页边距
	\geometry{left=2.5cm,right=2.5cm,top=3.2cm,bottom=2.8cm}
\usepackage{xeCJK,amsmath,paralist,enumerate,booktabs,multirow,graphicx,subfig,setspace,listings,lastpage,hyperref}
\usepackage{amsthm, amssymb, bm, color, framed, graphicx, hyperref, mathrsfs}
\usepackage{mathrsfs}  
	\setlength{\parindent}{2em}
	\lstset{language=Matlab}%
\usepackage{fancyhdr}
\usepackage{graphicx}
\usepackage{listings}
\usepackage{xcolor}
\usepackage{float}

\definecolor{mKeyword}{RGB}{0,0,255}          % bule
\definecolor{mString}{RGB}{160,32,240}        % purple
\definecolor{mComment}{RGB}{34,139,34}        % green
\definecolor{mNumber}{RGB}{128,128,128} 

\lstdefinestyle {njulisting} {
	basewidth = 0.5 em,
	lineskip = 3 pt,
	basicstyle = \small\ttfamily,
	% keywordstyle = \bfseries,
	commentstyle = \itshape\color{gray}, 
	basicstyle=\small\ttfamily,
	keywordstyle={\color{mKeyword}},     % sets color for keywords
	stringstyle={\color{mString}},       % sets color for strings
	commentstyle={\color{mComment}},     % sets color for comments
	numberstyle=\tiny\color{mNumber},
	numbers = left,
	captionpos = t,
	breaklines = true,
	xleftmargin = 2 em,
	xrightmargin = 2 em,
	frame=tlrb,
	tabsize=4
}

\lstset{
style = njulisting, % 调用上述样式 
flexiblecolumns % 允许调整字符宽度
}

\pagestyle{fancy}
\lhead{\textsc{Foundation of Computing System}}
\rhead{\textsc{Nanjing University}}
\cfoot{\thepage}
\renewcommand{\headrulewidth}{0.4pt}
\renewcommand{\theenumi}{(\arabic{enumi})}


\definecolor{shadecolor}{RGB}{241, 241, 255}

\newcommand{\problemname}{待定义}
\newenvironment{problem}{\begin{shaded}\par\noindent\textbf{题目\  \problemname}}{\end{shaded}\par}
\newenvironment{solution}{\par\noindent\textbf{解答}\ }{\par}
\newenvironment{note}{\par\noindent\textbf{题目 \problemname 的注记}}{\par}

\begin{document}

\begin{center}
\LARGE\textbf{第十五章习题参考答案}
\end{center}

{\kaishu 包含题目:习题$15.1$、$15.2$、$15.3$、$15.7$和$15.9$}


% \begin{figure}[H]
% 	\centering
% 	\includegraphics[scale=0.3]{img/}
% \end{figure}

% \lstset{language=C}
% 	\begin{lstlisting}
	
% 	\end{lstlisting}



\renewcommand{\problemname}{15.1}
\begin{problem}
	如下程序的输出是什么?
	\begin{lstlisting}[language=C]
#include <stdio.h> 

int Sub (int x, int y); 

int main () { 
	int x = 2; 
	int y = 3; 
	int z; 
  
	z = Sub (x, y); 
	x = Sub (y, z); 
	y = Sub (x, y); 
	printf ("%d %d %d", x, y, z); 
} 

int Sub (int y, int x) { 
	return y - x; 
}
	\end{lstlisting}
\end{problem}

\begin{solution}
	该程序的输出为4 1 $-1$
\end{solution}


\renewcommand{\problemname}{15.2}
\begin{problem}
	如下程序的输出是什么?
	\begin{lstlisting}[language=C]
#include <stdio.h>

int Multiply (int x, int y); 
int z = 4; 

int main () { 
	int x = 2; 
	int y = 3; 
	int z; 
	 
	z = Multiply (x, y); 
	x = Multiply (y, z); 
	y = Multiply (x, z); 
	printf ("%d %d %d \n ", z, x, y); 
}  
   
int Multiply (int x, int y)	{ 
	return x * y * z; 
} 		   
	\end{lstlisting}
\end{problem}

\begin{solution}
	该程序的输出为24 288 27648[换行]
\end{solution}


\renewcommand{\problemname}{15.3}
\begin{problem}
	如下程序的输出是什么?
	\begin{lstlisting}[language=C]
#include <stdio.h>

void Swap (int x, int y);

int main(){
	int x = 1;
	int y = 2;

	printf ("x=%d, y=%d\n", x, y);

	Swap(x, y);
	printf ("x=%d, y=%d\n", x, y);
}

void Swap(int x, int y){
	int temp;
	temp = x;
	x = y;
	y = temp;
	printf ("x=%d, y=%d\n", x, y);
}
	\end{lstlisting}
\end{problem}

\begin{solution}
	该程序的输出为:

	\verb|x=1, y=2|

	\verb|x=2, y=1|

	\verb|x=1, y=2|[换行]
\end{solution}


\renewcommand{\problemname}{15.7}
\begin{problem}
	使用DLX的C编译器对函数ToUpper进行编译,请写出DLX汇编代码。
	\begin{lstlisting}[language=C]
/*ToUpper函数*/
/*如果参数是小写字母*/
/*返回其大写字母的ASCII码*/ 
char ToUpper (char inchar) { 
	char outchar; 
	if ('a' <= inchar && inchar <= 'z') 
		outchar = inchar - ('a' - 'A'); 
	else 
		outchar = inchar; 
	return outchar; 
} 
	\end{lstlisting}
\end{problem}

\begin{solution}
	\begin{lstlisting}
ToUppeR:	SLTI 	R1, R4, #97 ; R4 saves inchar
			BNEZ 	R1, Else
			SLEI 	R1, R4, #122
			BEQZ 	R1, Else
			SUBI 	R2, R4, #32
			J 		Done
Else:
			ADDI 	R2, R4, #0 ; R2 saves outchar
			J 		Done
Done:
			ret
	\end{lstlisting}
\end{solution}


\renewcommand{\problemname}{15.9}
\begin{problem}
	对于如下C代码,存在bug,请找出并修复:
	\begin{lstlisting}[language=C]
#include <stdio.h>

int main() { 
	int x = 1; 
	int y = 2; 
	x = Func1 (x, y); 
	y = Func2 (y); 
	printf ("x = %d y = %d\n", x, y); 
} 

int Func1 (int x) { 
	return x + y; 
} 

int Func2 (int x) { 
	int y; 
	return x - y; 
} 
	\end{lstlisting}
\end{problem}

\begin{solution}
	\begin{enumerate}[1.]
		\item 函数未声明:
			需要在 \verb|main| 之前添加 \verb|Func1| 和 \verb|Func2| 的声明
			\begin{lstlisting}[language=C]
int Func1 (int);
int Func2 (int);				
			\end{lstlisting}

	   	\item 函数 \verb|Func1| 的参数个数不匹配
		\item 函数用到了未声明的变量:修改函数 \verb|Func1| 的参数结构,同时更新其声明。
		\begin{lstlisting}[language=C]
int Func1 (int, int);
…
int Func1 (int x, int y){…}			
		\end{lstlisting}
		\item 函数的局部变量没有初始化。(运行时错误)
		局部变量使用前最好都要初始化,防止读取为初始化的局部变量。
		\begin{lstlisting}[language=C]
int y = 0;
		\end{lstlisting}
	\end{enumerate}
	修改后的代码如下:
	\begin{lstlisting}[language=C]
#include <stdio.h> 

int Func1 (int, int);
int Func2 (int);

int main() { 
	int x = 1; 
	int y = 2; 
	x = Func1 (x, y); 
	y = Func2 (y); 
	printf ("x = %d y = %d\n", x, y); 
} 

int Func1 (int x, int y) { 
	return x + y; 
} 

int Func2 (int x) { 
	int y = 0; 
	return x - y; 
} 
	\end{lstlisting}
\end{solution}

\end{document}
