\documentclass[10pt,a4paper,UTF8]{ctexart}
\usepackage{geometry}%用于设置上下左右页边距
	\geometry{left=2.5cm,right=2.5cm,top=3.2cm,bottom=2.8cm}
\usepackage{xeCJK,amsmath,paralist,enumerate,booktabs,multirow,graphicx,subfig,setspace,listings,lastpage,hyperref}
\usepackage{amsthm, amssymb, bm, color, framed, graphicx, hyperref, mathrsfs}
\usepackage{mathrsfs}  
	\setlength{\parindent}{2em}
	\lstset{language=Matlab}%
\usepackage{fancyhdr}
\usepackage{listings}
\usepackage{xcolor}

\definecolor{mKeyword}{RGB}{0,0,255}          % bule
\definecolor{mString}{RGB}{160,32,240}        % purple
\definecolor{mComment}{RGB}{34,139,34}        % green
\definecolor{mNumber}{RGB}{128,128,128} 

\lstdefinestyle {njulisting} {
	basewidth = 0.5 em,
	lineskip = 3 pt,
	basicstyle = \small\ttfamily,
	% keywordstyle = \bfseries,
	commentstyle = \itshape\color{gray}, 
	basicstyle=\small\ttfamily,
	keywordstyle={\color{mKeyword}},     % sets color for keywords
	stringstyle={\color{mString}},       % sets color for strings
	commentstyle={\color{mComment}},     % sets color for comments
	numberstyle=\tiny\color{mNumber},
	numbers = left,
	captionpos = t,
	breaklines = true,
	xleftmargin = 2 em,
	xrightmargin = 2 em,
	frame=tlrb,
	tabsize=4
}

\lstset{
style = njulisting, % 调用上述样式 
flexiblecolumns % 允许调整字符宽度
}

\pagestyle{fancy}
\lhead{\textsc{Foundation of Computing System}}
\rhead{\textsc{Nanjing University}}
\cfoot{\thepage}
\renewcommand{\headrulewidth}{0.4pt}
\renewcommand{\theenumi}{(\arabic{enumi})}


\definecolor{shadecolor}{RGB}{241, 241, 255}

\newcommand{\problemname}{待定义}
\newenvironment{problem}{\begin{shaded}\par\noindent\textbf{题目\  \problemname}}{\end{shaded}\par}
\newenvironment{solution}{\par\noindent\textbf{解答}\ }{\par}
\newenvironment{note}{\par\noindent\textbf{题目 \problemname 的注记}}{\par}

\begin{document}

\begin{center}
\LARGE\textbf{第六章习题参考答案}
\end{center}

{\kaishu 包含题目:习题$6.3-6.5$以及$6.7-6.17$}

\renewcommand{\problemname}{6.3}
\begin{problem}
	将下列二进制数转化为十进制数,假设此二进制数分别为原码、反码和补码整数。
	\begin{enumerate}[(1)]
		\item 0111
		\item 1110
		\item 11111111
		\item 10000000
	\end{enumerate}
\end{problem}

\begin{solution}
	\begin{enumerate}[(1)]
		\item 若是原码,则为7;若是反码,则为7;若是补码,则为7。
		\item 若是原码,则为$-6$;若是反码,则为$-1$;若是补码,则为$-2$。
		\item 若是原码,则为$-127$;若是反码,则为0;若是补码,则为$-1$。
		\item 若是原码,则为$-0$;若是反码,则为$-127$;若是补码,则为$-128$。
	\end{enumerate}
\end{solution}


\renewcommand{\problemname}{6.4}
\begin{problem}
	将下列十进制数分别转化为8位二进制原码、反码和补码整数。
	\begin{enumerate}[(1)]
		\item $-86$
		\item 85
		\item $-127$
		\item 127
	\end{enumerate}
\end{problem}

\begin{solution}
	\begin{enumerate}[(1)]
		\item 原码1101 0110,反码1010 1001,补码1010 1010。
		\item 原码0101 0101,反码0101 0101,补码0101 0101。
		\item 原码1111 1111,反码1000 0000,补码1000 0001。
		\item 原码0111 1111,反码0111 1111,补码0111 1111。
	\end{enumerate}
	
\end{solution}


\renewcommand{\problemname}{6.5}
\begin{problem}
	如果二进制补码整数最后一位是0,表明该数是偶数,如果最后两位是00,则表明该数有什么特点?
\end{problem}

\begin{solution}
	能被4整除
\end{solution}


\renewcommand{\problemname}{6.7}
\begin{problem}
	对于一个二进制数,如果向右移一位,则意味着进行了什么运算?
\end{problem}

\begin{solution}
	除以2
\end{solution}


\renewcommand{\problemname}{6.8}
\begin{problem}
	做下列二进制补码整数加法运算,给出十进制形式的结果,并判断是否产生溢出。
	\begin{enumerate}[(1)]
		\item 1101 + 01010101
		\item 0111 + 0101
		\item 11111111 + 01
		\item 01 + 1110
		\item 0111 + 0001
		\item 1000 + 11
		\item 1100 + 00110011
		\item 1010 + 101
	\end{enumerate}
\end{problem}

\begin{solution}
	\begin{enumerate}[(1)]
		\item $1111 1101+0101 0101=(1)0101 0010=82$,没有溢出。
		\item $0111+ 0101=1101=-3$,溢出。
		\item $1111 1111+0000 0001=(1)0000 0000=0$,没有溢出。
		\item $0001+ 1110=1111=-1$,没有溢出。
		\item $0111+0001=1000=-8$,溢出。
		\item $1000+1111=(1)0111=7$,溢出。
		\item $1111 1100+0011 0011=(1)0010 1111=47$,没有溢出。
		\item $1010+1101=(1)0111=7$,溢出。
	\end{enumerate}
\end{solution}


\renewcommand{\problemname}{6.9}
\begin{problem}
	做下列二进制数逻辑运算,结果以二进制形式给出。
	\begin{enumerate}[(1)]
		\item 11001100 AND 01010101
		\item (1100 AND 0101) AND 1101
		\item 1100 AND (0101 AND 1101)
		\item 11001100 OR 01010101
		\item (1100 OR 0101) OR 1101
		\item 1100 OR(0101 OR 1101)
		\item NOT (NOT 1011)
		\item 1101 XOR 0101
		\item NOT( (NOT 1101) OR (NOT 0101) )
		\item NOT( (NOT 1101) AND (NOT 0101) )
		\item ( (NOT 1101) AND 0101) OR (1101 AND (NOT 0101) )
	\end{enumerate}
\end{problem}

\begin{solution}
	\begin{enumerate}[(1)]
		\item 0101 0111
		\item 0000
		\item 0000
		\item 1101 0111
		\item 1111
		\item 1111
		\item 0111
		\item 0111
		\item 1101
		\item 0110
	\end{enumerate}
	
\end{solution}


\renewcommand{\problemname}{6.10}
\begin{problem}
	给出下列十进制数的 IEEE 浮点数表示形式(32位),并将这些数的 IEEE 浮点数转换为十六进制表示。
	\begin{enumerate}[(1)]
		\item $32.9375$
		\item $-32\frac{45}{128}$
		\item $-2^{-140}$
		\item 65536
	\end{enumerate}
\end{problem}

\begin{solution}
	\begin{enumerate}[(1)]
		\item 0 1000 0100 0000 0111 1000 0000 0000 000,十六进制为0x4203C000
		\item 1 1000 0100 0000 0010 1101 0000 0000 000,十六进制为0xC2016800
		\item 1 0000 0000 0000 0000 0000 0100 0000 000,十六进制为0x80000200
		\item 0 1000 1111 0000 0000 0000 0000 0000 000,十六进制为0x47800000
	\end{enumerate}
\end{solution}


\renewcommand{\problemname}{6.11}
\begin{problem}
	给出下列 IEEE 浮点数的十进制数表示形式。
	\begin{enumerate}[(1)]
		\item 0 00000001 00000000000000000000000
		\item 0 00000000 0000000010000000000000
		\item 1 11111011 00000000000000000000000
		\item 1 10000001 10101000000000000000000
		\item 0 01111101 01010100000000000000000
	\end{enumerate}
\end{problem}

\begin{solution}
	\begin{enumerate}[(1)]
		\item $2^{-126}$
		\item $2^{-136}$ 
		\item $-2^{124}$
		\item $-6.625$
		\item $(0.01010101)_2=0.33203125$​
	\end{enumerate}
\end{solution}


\renewcommand{\problemname}{6.12}
\begin{problem}
	将下列16位的二进制补码整数的十六进制数转換为十进制数。
	\begin{enumerate}[(1)]
		\item x8000
		\item x7FFF
		\item x1234
		\item xABCD
	\end{enumerate}
\end{problem}

\begin{solution}
	\begin{enumerate}[(1)]
		\item 1000 0000 0000 0000 $=-2^{15}$
		\item 0111 1111 1111 1111 $=2^{15}-1$
		\item 0001 0010 0011 0100 $=4660$
		\item 1010 1011 1100 1101 $=-21555$
	\end{enumerate}
\end{solution}


\renewcommand{\problemname}{6.13}
\begin{problem}
	将下列十进制数转换为16位的二进制补码整数的十六进制表示。
	\begin{enumerate}[(1)]
		\item $-86$
		\item 85
		\item $-127$
		\item 127
	\end{enumerate}
\end{problem}

\begin{solution}
	\begin{enumerate}[(1)]
		\item 1111 1111 1010 1010,十六进制为0xFFAA
		\item 0000 0000 0101 0101,十六进制为0x0055
		\item 1111 1111 1000 0001,十六进制为0xFF81
		\item 0000 0000 0111 1111,十六进制为0x007F
	\end{enumerate}
	
\end{solution}


\renewcommand{\problemname}{6.14}
\begin{problem}
	如下代码将分别输出哪此内容?
	\begin{enumerate}[(1)]
		\item \verb|printf ("%c\n", 13 + 'A');|
		\item \verb|printf ("%x\n, 130);|
	\end{enumerate}
\end{problem}

\begin{solution}
	\begin{enumerate}[(1)]
		\item N(换行)
		\item 82(换行)
	\end{enumerate}
\end{solution}


\renewcommand{\problemname}{6.15}
\begin{problem}
	解释如下代码段的作用。
	\lstset{language=C}
	\begin{lstlisting}
char nextChar;
int x;

scanf ("%c", &nextChar);
printf ("%d\n", nextChar);

scanf ("%d", &x);
printf ("%c\n", x);		
	\end{lstlisting}
\end{problem}

\begin{solution}
	读入一个字符,并输出其ASCII码值;读入一个数字,并输出ASCII码值为该数字的相应字符。
\end{solution}


\renewcommand{\problemname}{6.16}
\begin{problem}
	描述如下代码段的作用及输出。
	\lstset{language=C}
	\begin{lstlisting}
int i, j;
int count = 0;

scanf ("%d", &i);
for (j = 0; j < 32; j++){
	if (i & (1 << j)){
		count++;
	}
}

printf("%d\n", count);
	\end{lstlisting}
\end{problem}

\begin{solution}
	输出输入数字二进制值中的1的个数。
\end{solution}


\renewcommand{\problemname}{6.17}
\begin{problem}
	如下代码是否会造成无限循环?
	\lstset{language=C}
	\begin{lstlisting}
int i = 1;
while (i > 0)
	i++;
	\end{lstlisting}
\end{problem}

\begin{solution}
	不会,当\verb|i|的值增大到整型的最大值时,再增加\verb|i|的值会发生负溢出,结束循环。
\end{solution}

\end{document}
