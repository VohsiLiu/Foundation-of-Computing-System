\documentclass[10pt,a4paper,UTF8]{ctexart}
\usepackage{geometry}%用于设置上下左右页边距
	\geometry{left=2.5cm,right=2.5cm,top=3.2cm,bottom=2.8cm}
\usepackage{xeCJK,amsmath,paralist,enumerate,booktabs,multirow,graphicx,subfig,setspace,listings,lastpage,hyperref}
\usepackage{amsthm, amssymb, bm, color, framed, graphicx, hyperref, mathrsfs}
\usepackage{mathrsfs}  
	\setlength{\parindent}{2em}
	\lstset{language=Matlab}%
\usepackage{fancyhdr}
\usepackage{graphicx}
\usepackage{listings}
\usepackage{xcolor}
\usepackage{float}

\definecolor{mKeyword}{RGB}{0,0,255}          % bule
\definecolor{mString}{RGB}{160,32,240}        % purple
\definecolor{mComment}{RGB}{34,139,34}        % green
\definecolor{mNumber}{RGB}{128,128,128} 

\lstdefinestyle {njulisting} {
	basewidth = 0.5 em,
	lineskip = 3 pt,
	basicstyle = \small\ttfamily,
	% keywordstyle = \bfseries,
	commentstyle = \itshape\color{gray}, 
	basicstyle=\small\ttfamily,
	keywordstyle={\color{mKeyword}},     % sets color for keywords
	stringstyle={\color{mString}},       % sets color for strings
	commentstyle={\color{mComment}},     % sets color for comments
	numberstyle=\tiny\color{mNumber},
	numbers = left,
	captionpos = t,
	breaklines = true,
	xleftmargin = 2 em,
	xrightmargin = 2 em,
	frame=tlrb,
	tabsize=4
}

\lstset{
style = njulisting, % 调用上述样式 
flexiblecolumns % 允许调整字符宽度
}

\pagestyle{fancy}
\lhead{\textsc{Foundation of Computing System}}
\rhead{\textsc{Nanjing University}}
\cfoot{\thepage}
\renewcommand{\headrulewidth}{0.4pt}
\renewcommand{\theenumi}{(\arabic{enumi})}


\definecolor{shadecolor}{RGB}{241, 241, 255}

\newcommand{\problemname}{待定义}
\newenvironment{problem}{\begin{shaded}\par\noindent\textbf{题目\  \problemname}}{\end{shaded}\par}
\newenvironment{solution}{\par\noindent\textbf{解答}\ }{\par}
\newenvironment{note}{\par\noindent\textbf{题目 \problemname 的注记}}{\par}

\begin{document}

\begin{center}
\LARGE\textbf{第十四章习题参考答案}
\end{center}

{\kaishu 包含题目:习题$14.7-14.10$}


% \begin{figure}[H]
% 	\centering
% 	\includegraphics[scale=0.3]{img/}
% \end{figure}

% \lstset{language=C}
% 	\begin{lstlisting}
	
% 	\end{lstlisting}



\renewcommand{\problemname}{14.7}
\begin{problem}
	假设位于存储单元NUM中的值可以是任意一个大于2的32位的正整数。如下程序实现了什么?
	\begin{lstlisting}
			.data	x30000000 
NUM:		.word	#201 
RESULT:		.space	#4 
;		 
			.text	x40000000 
			.global	main 
main:		addi	r1, r0, #0 
			addi	r2, r0, #2 
			lw		r3, NUM(r0) 
;
LOOP:		jal		MOD 
			beqz	r6, EXIT0 
			addi	r2, r2, #1 
			seq		r6, r2, r3 
			bnez	r6, EXIT1 
			j		LOOP 
; 
EXIT1:		addi	r1, r1, #1 
EXIT0:		sw		RESULT(r0), r1 
			trap	x00 
;
MOD:		addi	r4, r3, #0 
AGAIN:		sub		r4, r4, r2 
			slei	r6, r4, #0 
			beqz	r6, AGAIN 
			jr		r31
	\end{lstlisting}
\end{problem}

\begin{solution}
	将上述汇编代码写成伪代码如下
	\begin{lstlisting}
r1 = 0; r2 = 2; r3 = NUM;
Do{
	MOD()
	If (r6 == 0) 
		break;
	r2 = r2 + 1;
	r6 = r2 == r3;
	If (r6 == 1) {
		r1 = r1 + 1; 
		break;
	}
} while(1);
RESULT = r1;
MOD(){
	r4 = r3; 
	do{
		r4 = r4 - r2;
		r6 = r4 <= 0;
	} while(r6 == 0);
}
	\end{lstlisting}
	由于调用子例程MOD结束后r6始终是1,所以不会从EXIT0直接退出,实现了r1 = 1。
\end{solution}


\renewcommand{\problemname}{14.8}
\begin{problem}
	假设从存储单元DATA开始存储的10个值可以是任意的10个32位的整数。如下程序实现了什么?
	\begin{lstlisting}
			.data	x30000000 
DATA:		.word	#3, #14, #35, #47, #5, #20, #12, #14, #6, #22 
SaveR31:	.space	#4 
;		 
			.text	x40000000 
			.global	main 
main:		addi	r1, r0, DATA 
			addi	r2, r0, #9 
OutLoop:	beqz	r2, EXIT 
			addi	r3, r2, #0 
InnerLoop:	lw		r4, 0(r1) 
			lw		r5, 4(r1) 
			jal		CMP 
			addi	r1, r1, #4 
			subi	r3, r3, #1 
			bnez	r3, InnerLoop 
			slli	r6, r2, #2 
			sub		r1, r1, r6 
			subi	r2, r2, #1 
			j		OutLoop	 
EXIT:		trap	x00 
; 
CMP:		sw		SaveR31(r0), r31 
			slt		r6, r4, r5 
			bnez	r6, Return 
			jal		SWAP 
Return:		lw		r31, SaveR31(r0) 
			ret 
; 
SWAP:		sw		4(r1), r4 
			sw		0(r1), r5 
			ret 
	\end{lstlisting}
\end{problem}

\begin{solution}
	将上述代码翻译如下
	\begin{lstlisting}
r1 = [DATA]; r2 = 9;
While (r2 != 0){
	r3 = r2; 
	Do{
		r4 = r1[0];
		r5 = r1[1];
		CMP();
		r1 = r1 + 4;
		r3 = r3 - 1;
	} while(r3 != 0);
	r6 = r2 * 4;
	r1 = r1 - r6;
	r2 = r2 - 1;
}
CMP(){
	r6 = r4 < r5; 
	if (r6 == 0){
		SWAP();
	}
}
SWAP(){
	r1[1] = r4;
	r1[0] = r5;
}
	\end{lstlisting}
	比较相邻两个数,前者不小于后者就交换,每次OutLoop选出一个最大数,实现排序。
\end{solution}


\renewcommand{\problemname}{14.9}
\begin{problem}
	如下程序将存储单元NUM中的数值显示在屏幕上,但是不能正常工作。请找出原因并修复。
	\begin{lstlisting}
		.data	x30000000 
NUM:	.word	#8 
; 
		.text	x40000000 
		.global	main 
main:	jal		SubA 
		trap	x07 
		trap	x00 
; 
SubA:	lw		r1, NUM(r0) 
		jal		SubB 
		ret
; 
SubB:	addi	r4, r1, x30 
		ret 
	\end{lstlisting}
\end{problem}

\begin{solution}
	嵌套子程序中上一层必须保存r31,防止被下一层修改。
	可以通过在调用 \verb|jal SubB| 之前保存r31,调用之后恢复r31来修复。
\end{solution}

\renewcommand{\problemname}{14.10}
\begin{problem}
	如下程序实现了什么?
	\begin{lstlisting}
		.data	x30000000 
STACK:	.space	#100 
PROMPT:	.asciiz	"Please enter your string: " 
; 
		.text	x40000000 
		.global	main 
main:	addi	r4, r0, PROMPT 
		trap	x08 
		addi	r3, r0, STACK 
		addi	r5, r3, #100	
		addi	r29, r5, #0 
Input:	seq		r2, r29, r3 
		bnez	r2, Output 
		trap	x06 
		trap	x07 
		seqi	r2, r4, #10 
		bnez	r2, Output 
		jal		PUSH 
		j		Input 
; 
Output:	seq		r2, r29, r5 
		bnez	r2, DONE 
		jal		POP 
		trap	x07 
		j		Output
; 
DONE:	trap	x00 
; 
PUSH:	subi	r29, r29, #4 
		sw		0(r29), r4 
		ret 
; 
POP:	lw		r4, 0(r29) 
		addi	r29, r29, #4 
		ret 
	\end{lstlisting}
\end{problem}

\begin{solution}
	将上述代码翻译如下
	\begin{lstlisting}
printf("Please enter your string:");
r3 = [STACK]; r5 = r3 + 100; r29 = r5;
While (1){
	r2 = r29 == r3; 
	if (r2 == 1) 
		Output();
	scanf(); 
	printf(); 
	r2 = r4 == 10; 
	if (r2 == 1) 
		Output(); 
	PUSH();
}
Output(){
	r2 = r29 == r5;
	if (r2 == 1) 
		exit; 
	POP(); 
	printf(); 
	Output();
}
PUSH(){
	r29 = r29 - 4; 
	r29[0] = r4;
}
POP(){
	r4 = r29[0];
	r29 = r29 + 4;
}
	\end{lstlisting}
	带回显地输入以换行符为结束标记的字符串(最长25),再以反序输出到屏幕上。
\end{solution}

\end{document}
