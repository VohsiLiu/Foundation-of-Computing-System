\documentclass[10pt,a4paper,UTF8]{ctexart}
\usepackage{geometry}%用于设置上下左右页边距
	\geometry{left=2.5cm,right=2.5cm,top=3.2cm,bottom=2.8cm}
\usepackage{xeCJK,amsmath,paralist,enumerate,booktabs,multirow,graphicx,subfig,setspace,listings,lastpage,hyperref}
\usepackage{amsthm, amssymb, bm, color, framed, graphicx, hyperref, mathrsfs}
\usepackage{mathrsfs}  
	\setlength{\parindent}{2em}
	\lstset{language=Matlab}%
\usepackage{fancyhdr}
\usepackage{listings}
\usepackage{xcolor}

\definecolor{mKeyword}{RGB}{0,0,255}          % bule
\definecolor{mString}{RGB}{160,32,240}        % purple
\definecolor{mComment}{RGB}{34,139,34}        % green
\definecolor{mNumber}{RGB}{128,128,128} 

\lstdefinestyle {njulisting} {
	basewidth = 0.5 em,
	lineskip = 3 pt,
	basicstyle = \small\ttfamily,
	% keywordstyle = \bfseries,
	%commentstyle = \itshape\color{gray}, 
	basicstyle=\small\ttfamily,
	keywordstyle={\color{mKeyword}},     % sets color for keywords
	stringstyle={\color{mString}},       % sets color for strings
	commentstyle={\color{mComment}},     % sets color for comments
	numberstyle=\tiny\color{mNumber},
	numbers = left,
	captionpos = t,
	breaklines = true,
	xleftmargin = 2 em,
	xrightmargin = 2 em,
	frame=tlrb,
	tabsize=4
}

\lstset{
style = njulisting, % 调用上述样式 
flexiblecolumns % 允许调整字符宽度
}

\pagestyle{fancy}
\lhead{\textsc{Foundation of Computing System}}
\rhead{\textsc{Nanjing University}}
\cfoot{\thepage}
\renewcommand{\headrulewidth}{0.4pt}
\renewcommand{\theenumi}{(\arabic{enumi})}


\definecolor{shadecolor}{RGB}{241, 241, 255}

\newcommand{\problemname}{待定义}
\newenvironment{problem}{\begin{shaded}\par\noindent\textbf{题目\  \problemname}}{\end{shaded}\par}
\newenvironment{solution}{\par\noindent\textbf{解答}\ }{\par}
\newenvironment{note}{\par\noindent\textbf{题目 \problemname 的注记}}{\par}

\begin{document}

\begin{center}
\LARGE\textbf{第一章习题参考答案}
\end{center}

{\kaishu 包含题目:习题1.2、1.6、1.7、1.8和1.9}


\renewcommand{\problemname}{1.2}
\begin{problem}
	有两台计算机 A 和 B,A有乘法指令,而 B没有;两者都有加法和减法指令;在其他方面两者都相同。
	那么,在A和B中,哪台计算机可以解决更多的问题?
\end{problem}

\begin{solution}
	所有的计算机(无论大还是小,快还慢,昂贵还是便宜),如果给予足够的时间和足够的存储器,
	都可以做相同的计算。换句话说,所有的计算机都能做几乎完全相同的事情,只是计算速度上有差别。
\end{solution}


\renewcommand{\problemname}{1.6}
\begin{problem}
	给出如下问题的算法。
	\begin{enumerate}[(1)]
		\item 计算$1+2+3+4+5+6+7+8+9+10$
		\item 判定 $2010-2500$ 年中的某一年是否为闰年
		\item 对一个大于或等于3的正整数,判断它是否为素数(质数)
	\end{enumerate}
\end{problem}

\begin{solution}
	以下参考解法均采用C语言代码描述,也可以用伪代码或自然语言描述等
	\begin{enumerate}[(1)]
		\item \ 
\lstset{language=C}
\begin{lstlisting}
int sum (int n){
	int result = 0;
	for(int i = 1; i <= n; i++){
		sum += i;
	}
	return result;
}
\end{lstlisting}

		\item \ 
\lstset{language=C}
\begin{lstlisting}
int isLeapYear (int year){
	return (year % 4 == 0) && (year % 100 != 0) || (year % 400 == 0);
}
\end{lstlisting}

		\item \ 
\lstset{language=C}
\begin{lstlisting}
bool isPrime (int num){
	for (int i= 2; i <= sqrt(num); i++){
	if (num % i == 0)
		return false;
	}
	return true;
}
\end{lstlisting}
	\end{enumerate}
\end{solution}


\renewcommand{\problemname}{1.7}
\begin{problem}
	当将计算机升级(如更换CPU)后,原来的软件(如操作系统)还能够使用吗?
\end{problem}

\begin{solution}
	言之有理即可。
	\begin{itemize}
		\item 能用。计算机升级升的是硬件,软件存储在磁盘中,
		只要升级的硬件提供与原有硬件相同的工作方式和功能,软件还是能够正常工作的。
		\item 不能用。计算机中的硬件升级前与升级后所需要的驱动可能不一样,
		软件无法通过原有方法调用底层硬件,因此也就不能使用了。
	\end{itemize}
\end{solution}


\renewcommand{\problemname}{1.8}
\begin{problem}
	人们购买的软件通常是以什么语言编写的?是高级语言还是与目标机器ISA兼容的机器语言?
\end{problem}

\begin{solution}
	软件可能存在的方式是多种多样的:源代码或目标代码。源代码包括汇编语言、
	3GL和4GL、经验知识等。目标代码包括机器语言、解释型源代码等。
	具体可参考:
	\href{http://www.rogerclarke.com/SOS/PaperLiaby.html}{http://www.rogerclarke.com/SOS/PaperLiaby.html}
\end{solution}


\renewcommand{\problemname}{1.9}
\begin{problem}
	关于计算系统的每个抽象层次,分别举出两个以上的例子进行说明。
\end{problem}

\begin{solution}
	计算系统的抽象层次:问题、算法、程序、操作系统、指令集系统、微处理器、逻辑电路、元件。
\end{solution}

\end{document}
