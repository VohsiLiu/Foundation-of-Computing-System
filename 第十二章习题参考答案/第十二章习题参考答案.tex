\documentclass[10pt,a4paper,UTF8]{ctexart}
\usepackage{geometry}%用于设置上下左右页边距
	\geometry{left=2.5cm,right=2.5cm,top=3.2cm,bottom=2.8cm}
\usepackage{xeCJK,amsmath,paralist,enumerate,booktabs,multirow,graphicx,subfig,setspace,listings,lastpage,hyperref}
\usepackage{amsthm, amssymb, bm, color, framed, graphicx, hyperref, mathrsfs}
\usepackage{mathrsfs}  
	\setlength{\parindent}{2em}
	\lstset{language=Matlab}%
\usepackage{fancyhdr}
\usepackage{graphicx}
\usepackage{listings}
\usepackage{xcolor}
\usepackage{float}

\definecolor{mKeyword}{RGB}{0,0,255}          % bule
\definecolor{mString}{RGB}{160,32,240}        % purple
\definecolor{mComment}{RGB}{34,139,34}        % green
\definecolor{mNumber}{RGB}{128,128,128} 

\lstdefinestyle {njulisting} {
	basewidth = 0.5 em,
	lineskip = 3 pt,
	basicstyle = \small\ttfamily,
	% keywordstyle = \bfseries,
	commentstyle = \itshape\color{gray}, 
	basicstyle=\small\ttfamily,
	keywordstyle={\color{mKeyword}},     % sets color for keywords
	stringstyle={\color{mString}},       % sets color for strings
	commentstyle={\color{mComment}},     % sets color for comments
	numberstyle=\tiny\color{mNumber},
	numbers = left,
	captionpos = t,
	breaklines = true,
	xleftmargin = 2 em,
	xrightmargin = 2 em,
	frame=tlrb,
	tabsize=4
}

\lstset{
style = njulisting, % 调用上述样式 
flexiblecolumns % 允许调整字符宽度
}

\pagestyle{fancy}
\lhead{\textsc{Foundation of Computing System}}
\rhead{\textsc{Nanjing University}}
\cfoot{\thepage}
\renewcommand{\headrulewidth}{0.4pt}
\renewcommand{\theenumi}{(\arabic{enumi})}


\definecolor{shadecolor}{RGB}{241, 241, 255}

\newcommand{\problemname}{待定义}
\newenvironment{problem}{\begin{shaded}\par\noindent\textbf{题目\  \problemname}}{\end{shaded}\par}
\newenvironment{solution}{\par\noindent\textbf{解答}\ }{\par}
\newenvironment{note}{\par\noindent\textbf{题目 \problemname 的注记}}{\par}

\begin{document}

\begin{center}
\LARGE\textbf{第十二章习题参考答案}
\end{center}

{\kaishu 包含题目:习题$12.6-12.8$}


% \begin{figure}[H]
% 	\centering
% 	\includegraphics[scale=0.3]{img/}
% \end{figure}

% \lstset{language=C}
% 	\begin{lstlisting}
	
% 	\end{lstlisting}



\renewcommand{\problemname}{12.6}
\begin{problem}
	重新设计DLX的键盘输入处理程序:使用一个可读/写键盘数据和状态寄存器(KBDSR)
	代替KBDR和KBSR,且将内存地址 xFFFF 0010 $\sim$ xFFFF 0013 分配给 KBDSR。
	KBDSR 中包含与KBDR相同的数据,
	即从键盘输入的ASCII码。那么,输入例程只需检查 KBDSR 是否有非零值出现,
	然后读取这个值,当读出该值
	后,硬件电路将KBDSR清零。参考12.2节,给出新的输入例程。
\end{problem}

\begin{solution}
	\begin{lstlisting}
A:		.word	xFFFF0010	; KBDSR的起始地址
		…… 
		LW		R1, A(R0) 
START:	LW		R2, 0(R1)	; 测试是否有字符被输入
		ADNI	R3, R2, #-1 
		BEQZ	R3, START 
		LW		R4, 0(R1) 
		J		NEXT_TASK	; 执行下一个任务
	\end{lstlisting}
\end{solution}


\renewcommand{\problemname}{12.7}
\begin{problem}
	重新设计 DLX 的输入/输出处理程序:使用一个输入/输出状态寄存器(IOSR),代替KBSR 和 DSR。
	IOSR[1]是键盘设备就绪位,IOSR[0]是显示器设备就绪位,且将内存地址 xFFFF 0010 $\sim$ xFFFF 0013 分配给
	IOSR。参考12.2节和12.3节,给出新的输入例程和输出例程。
\end{problem}

\begin{solution}
	输入例程
	\begin{lstlisting}
A:		.word	xFFFF0010	; IOSR的起始地址
B:		.word	xFFFF0004	; KBDR的起始地址
…… 
		LW		R1, A(R0) 
START:	LW		R2, 0(R1) 	; 测试是否有字符被输入
		ANDI	R3, R2, #2 
		BEQZ	R3, START 
		LW		R1, B(R0) 
		LW		R4, 0(R1) 
		J		NEXT_TASK	; 执行下一个任务
	\end{lstlisting}
	输出例程
	\begin{lstlisting}
A:		.word	xFFFF0010	; IOSR的起始地址
B:		.word	xFFFF000C	; DDR的起始地址
…… 
		LW		R1, A(R0) 
START:	LW		R2, 0(R1) 	; 测试输出寄存器是否就绪
		ANDI	R3, R2, #1 
		BEQZ	R3, START 
		LW		R1, B(R0) 
		SW		0(R1), R4 
		J		NEXT_TASK	; 执行下一个任务
	\end{lstlisting}
\end{solution}


\renewcommand{\problemname}{12.8}
\begin{problem}
	如下DLX程序实现了什么?
	\begin{lstlisting}
		.data
DSR:	.word	xFFFF0008	 
DDR:	.word	xFFFF000C 
; 
		.text	
		ADDI	R1, R0, x61 
		ADDI	R2, R0, #26 
LOOP:	LW		R3, DSR(R0) 
START:	LW		R4, 0(R3) 
		ADDI	R5, R4, #1 
		BEQZ	R5, START	
		LW		R3, DDR(R0) 
		SW		0(R3), R1 
		ADDI	R1, R1, #1 
		SUBI	R2, R2, #1 
		BNEZ	R2, LOOP 
		J		NEXT_ TASK
	
	\end{lstlisting}
\end{problem}

\begin{solution}
	将上述程序写成伪代码
	\begin{lstlisting}[language=C]
r1 = 'a'
r2 = 26
do{
	print r1
	r1 = r1 + 1
	r2 = r2 - 1
}while (r2 != 0);
	\end{lstlisting}
	本程序实现了输出从“a”到“z”这26个字母。
\end{solution}

\end{document}
